\section{Chapter 0: Energy and Global Income Distribution}

\subsection{How is income distributed globally, and how does it relate to energy consumption?}
\solutionblock{If we divide the world into 4 groups \nicefrac{1}{7} would earn under 2\$, another \nicefrac{3}{7} between 2\$ and 8\$, another \nicefrac{2}{7} between 8\$ and 32\$ and the last \nicefrac{1}{7} would earn more than 32\$ a day. The energy consumption is distributed in a similar way.\\
In richer countries people eat more, drive more, fly more, and use more utilities, all leading to higher energy consumption.}
\subsection{Compute the primary energy consumption in a fully developed country per capita and day from:}
\subsubsection{a) Estimating a person's individual consumption (heating, electricity, car, etc.)}
\solutionblock{I roughly pay for about 1000 Kwh of electricity per year. That is about 3 Kwh == 3*(3600*1000) = 10.8 MJ per day.\\
Heating is roughly double to triple that, so about 25 MJ per day.\\
1 L of gasoline has about 32 MJ of energy and I drive about 10.000 km a year: 10.000 km / 5 L per 100 km = 2000 L of gasoline per year. That is 64 GJ per year or 175 MJ per day.\\
Clearly I'm driving a lot. With production of goods + transportation and other stuff I'd say I'm at about 300 MJ per day roughly 85 Kwh/day\\
\textbf{Note: 1 J = 1 Ws; 1 Kwh = 3600 * 1000 J = 3,6 MJ; 1 MJ = 0,277 Kwh}}

\subsubsection{b) From the macroeconomic perspective of a whole country}
\solutionblock{Primary energy consumption of Austria is about $1,4*10^18$ J/year which is (divided by 365* 10 Mio. [people]) about 380 MJ per person per day or about 105 Kwh per person per day.}

\subsection{Explain energy intensity}
\solutionblock{Energy intensity is a measure of the energy inefficiency of an economy. It is calculated as units of energy per unit of GDP (Gross Domestic Product) or some other measure of economic output. High energy intensities indicate a high price or cost of converting energy into GDP. (Wikipedia)\\
Depends on multiple factors: climate, energy mix and sectors in the economy of the given country (e.g. industry vs.  services vs. agriculture, etc.)}

\subsection{How do primary energy consumption and consumer electricity differ?}
\solutionblock{1) Primary energy consumption is the energy contained in the fuel (e.g. coal, oil, gas, uranium, etc.)\\
2) Consumer electricity is the energy that is delivered to the consumer (e.g. electricity from the wall socket).\\
As seen in the question above, the actual electricity consumption (bill) I gave was about 10\% of the primary energy consumption per person.\\
\textbf{Of the primary energy only about 20\% are actually converted to electricity and half of that is used by consumers (rest in industry, loss etc.)}}

\subsection{What's the energy mix in Austria?}
\solutionblock{See this \href{https://www.bmk.gv.at/dam/jcr:da4e9dfd-f51c-44b8-894c-9b049a8336cb/BMK_Energie_in_OE2023_barrierefrei.pdf}{[LINK]}\\
\begin{figure}[H]
    \centering
    \includegraphics[width=0.8\textwidth]{chapters/fig/0_energy_mix_comp.png}
    \caption{Energy mix in Austria}
    \label{fig:energy_mix_austria}
\end{figure}
So we have mostly oil and gas (~ 55\%) and then renewable energy sources (~ 35\%) and coal and waste + import (~ 10\%). 
}

\subsection{Given a number for reserves of a single fossil resource, compute:}
\subsubsection{a) What part of the energy mix it can contribute sustainably (1000 years)}
\solutionblock{}

\subsubsection{b) How long would it last at current consumption levels}
\solutionblock{}

\subsection{How much W/m\textsuperscript{2} can various energy sources produce? How would you compute it?}
\solutionblock{}

\subsection{Discuss whether renewables compete with arable land for food production. What about biofuels?}
\solutionblock{}

\subsection{Discuss the ongoing price-drop in solar, and what it means for other alternatives, such as future fusion energy}
\solutionblock{}

\subsection{Discuss the question of perceived and quantitative risk for the environment from various aspects of civilization (e.g., birds vs cats/wind turbines)}
\solutionblock{}

\subsection{Are there CO\textsubscript{2}-free energy sources? Why/why not?}
\solutionblock{}
