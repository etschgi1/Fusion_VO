\section{Video Notes: }
\subsection{JET}
JET is a large collaborative project between many European countries. It is located in the UK and is the largest tokamak in the world. It is a research facility and has no goal of producing energy. It is used to test new technologies and materials for future fusion reactors.\\

JET does research to give input to ITER and DEMO. 
The wall of JET was changed to a beryllium and tungsten wall to test the material for ITER. It is the only tokamak in the world that can be fueled with DT.\\
Less fuel (tritium) trapped in the new wall. \\
Electron density is increased in for tritium rich plasmas, but the erosion of the wall is also increased by tritium.\\
Due to the new wall some Beryllium is present in the plasma, but this can be used to heat the plasma.\\

JET doubled the previous record for fusion power with a tritium rich plasma 59 MJ. 

SUMMARY: D-T fusion in ITER like conditions tested, we know more about burning plasma + models validated. 
\subsection{NIF - Lawrence Livermore National Lab}
Lawson criteria in inertial confinement fusion (principle is implosion):\\
They use indirect drive with a hohlraum -> the hohlraum surface emits x-rays which then heat the fuel pellet.\\
Typical pressure in X-rays on capsule surface is about 100-200 Mbar (in DT 100-550 Gbar).\\
Alpha heating: the alpha particles produced by the fusion reaction heat the plasma.\\
Implosion velocity is about 350 km/s and after the implosion the plasma explodes (taking energy out)\\
A burning ICF plasma means that the alpha heating is higher than the energy loss (by explotion, brems loss and conduction loss).\\
Changed design to get the alpha heating to exceed the energy loss and get to ignite. 

\subsection{MIT Plasma Sciene and Fusion Center}
\textbf{20 Tesla superconducting magnets} \\
High field means high gain: 
HTS (High Temperature Superconductors) > 4K -> operating space is bigger by a factor of a 1000\\
Tested - simulated and then tested in the lab. Approximately double the field strength of current magnets.-> to build SPARC as a proof of concept.\\
SPARC proofs that high field magnets are possible and that fusion reactors can run for about 10 seconds per pulse. 

