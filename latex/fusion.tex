\documentclass[a4paper, 11pt, nenglish, parskip=half-]{scrartcl}

\title{Fusion Physics}
\subtitle{\href{https://github.com/etschgi1/Fusion_VO}{\underline{Newest version (link)}}}
\date{\today}
\author{Elias Wachmann \& David Obermaier}

\usepackage[margin=1in]{geometry}
\usepackage[utf8]{inputenc}
\usepackage[T1]{fontenc}
\usepackage{lmodern}
\usepackage{babel}
\usepackage{csquotes}
\usepackage{xurl}
\usepackage{amsmath, amssymb, amstext, mathtools}
\usepackage{icomma}
\usepackage[locale=DE, uncertainty-mode=separate]{siunitx}
\usepackage{physics}
\usepackage{derivative}
\usepackage{nicefrac}
\usepackage{pdfpages}
\usepackage{lastpage}
\usepackage{graphicx}
\usepackage{float}
\usepackage[version=4]{mhchem}
\usepackage{chemformula}
\usepackage{xcolor}
\usepackage{mdframed}
\usepackage[most]{tcolorbox}
\usepackage{tikz}
\usepackage[hidelinks,colorlinks]{hyperref}

\hypersetup{
    colorlinks = false,
    linkbordercolor = {white}
}

\usepackage[headsepline]{scrlayer-scrpage}
\pagestyle{scrheadings}
\setkomafont{pageheadfoot}{\normalfont}
\chead{Fusion Physics}
\ohead{\today}
\cfoot{\pagemark{} / \pageref*{LastPage}}

\usepackage{caption, subcaption}
\captionsetup[table]{name=Tabelle}
\captionsetup[figure]{name=Abbildung}
\captionsetup{format=plain, font=small, labelfont=bf, justification=centering}

\newcommand{\myparagraph}[1]{\paragraph{#1}\mbox{}\\}
\newcommand{\NA}{\ensuremath{\mathit{NA}}}
\renewcommand{\thesection}{\Alph{section}}
\newcounter{question}
\newcommand{\question}[1]{\stepcounter{question}\paragraph{Question \thequestion: #1}~}
\newcommand{\qref}[1]{\textbf{\hyperref[q:#1]{#1}}}
\newcommand{\aqref}[1]{\textbf{\hyperref[q:#1]{Question #1}}}
\renewcommand{\vec}[1]{\vb{#1}}

\definecolor{darkgreen}{rgb}{0.0, 0.5, 0.0}
\definecolor{lightgray}{rgb}{0.85, 0.85, 0.85}
\definecolor{darkgray}{rgb}{0.7, 0.7, 0.7}

\tcbset{
    solutionbox/.style={
        colback=lightgray,
        colframe=lightgray,
        boxrule=0pt,
        toprule=1pt,
        sharp corners,
        left=10pt,
        right=10pt,
        top=10pt,
        bottom=10pt,
        before upper={\tikz\node[draw, rounded corners, fill=darkgray, text=darkgreen, inner xsep=5pt, inner ysep=2pt] {Solution};\par\vspace{5pt}},
    },
    multisolutionbox/.style={
        colback=lightgray,
        colframe=lightgray,
        boxrule=0pt,
        toprule=1pt,
        sharp corners,
        left=10pt,
        right=10pt,
        top=10pt,
        bottom=10pt,
        breakable, % Allows the box to split across pages
        title={\tikz\node[draw, rounded corners, fill=darkgray, text=darkgreen, inner xsep=5pt, inner ysep=2pt] {Solution};\par\vspace{5pt}},
    }
}

% Single-page solution box
\newcommand{\solutionblock}[1]{
    \begin{tcolorbox}[solutionbox]
        #1
    \end{tcolorbox}
    \vspace{1em}
}

% Multi-page solution box
\newenvironment{multisolutionblock}{
    \begin{tcolorbox}[multisolutionbox]
}{
    \end{tcolorbox}
}

\begin{document}

\maketitle
\newpage
\tableofcontents
\newpage


\section{Chapter 0: Energy and Global Income Distribution}

\subsection{How is income distributed globally, and how does it relate to energy consumption?}
\solutionblock{}

\subsection{Compute the primary energy consumption in a fully developed country per capita and day from:}
\subsubsection{a) Estimating a person's individual consumption (heating, electricity, car, etc.)}
\solutionblock{}

\subsubsection{b) From the macroeconomic perspective of a whole country}
\solutionblock{}

\subsection{Explain energy intensity}
\solutionblock{}

\subsection{How do primary energy consumption and consumer electricity differ?}
\solutionblock{}

\subsection{What's the energy mix in Austria?}
\solutionblock{}

\subsection{Given a number for reserves of a single fossil resource, compute:}
\subsubsection{a) What part of the energy mix it can contribute sustainably (1000 years)}
\solutionblock{}

\subsubsection{b) How long would it last at current consumption levels}
\solutionblock{}

\subsection{How much W/m\textsuperscript{2} can various energy sources produce? How would you compute it?}
\solutionblock{}

\subsection{Discuss whether renewables compete with arable land for food production. What about biofuels?}
\solutionblock{}

\subsection{Discuss the ongoing price-drop in solar, and what it means for other alternatives, such as future fusion energy}
\solutionblock{}

\subsection{Discuss the question of perceived and quantitative risk for the environment from various aspects of civilization (e.g., birds vs cats/wind turbines)}
\solutionblock{}

\subsection{Are there CO\textsubscript{2}-free energy sources? Why/why not?}
\solutionblock{}

\newpage
\section{Chapter 1: Nuclear Energy and Fusion}

\subsection{Give a historical perspective of the controlled use of nuclear energy in general and fusion in particular.}
\solutionblock{In the early days scientists dreamed about changing Lead into Gold. With better knowledge of chemical reactions they at least tried but were doomed to fail from the beginning.\\ At the end of the 19th century scientists discovered that atoms are not the smallest particles and that they can be split. Especially the discovery of radiation and radioactive decay was a big step forward.\\ Looking at the sun, they wondered how it could burn for so long. The answer was nuclear fusion and Einstein gave us the world famous formula $E = m c^2$ in 1905, which relates energy and mass. Yet it was Francis Aston's discovery or rather precise measurement of the mass of Helium and Hydrogen that gave us the final piece of the puzzle. He measured that the mass of Helium was less than the mass of 4 Hydrogen atoms and postulated that the difference in mass was converted to energy by a nuclear reaction.\\ Only in the late 20s a complete understanding of the nuclear fusion process was achieved by also taking quantum mechanics into account (otherwise the sun would have been to cold for any reaction to take place).\\
In the late 30s Otto Hahn and Fritz Strassmann discovered nuclear fission of uranium by bombarding it with neutrons.

\textbf{Fission:} After the second world war first attempts were made to build a fusion reactor. The idea was to fuse deuterium and tritium together to form Helium. Deuterium is a Hydrogen Isotope naturally found in water and tritium was made using Lithium in so called breeder reactors. The US, UK and Soviet Union all started their classified fusion programs in the 50s. First the confinement was done using magnetic fields, this turned out to be very difficult. Later the idea of inertial confinement was born with the invention of the laser (1960). There a laser is used to insert energy into a pellet of fuel which then is only confined by its own inertia.\\
Both inertial and magnetic confinement fusion are still being researched today and are the most promising ways to achieve fusion in a viable way.\\}

\subsection{Draw and explain a schematic fusion power plant.}
\solutionblock{
\textbf{A note on Li:} To get tritium for the fusion reaction Lithium is used as a blanket around the plasma. The neutrons from the fusion reaction hit the Lithium and produce tritium.\\\\
Like a nuclear fission plant a fusion plant is also a thermal power plant using the heat from the fusion reaction to produce steam and drive a turbine.\\
\begin{figure}[H]
    \centering
    \includegraphics[width=0.8\textwidth]{chapters/fig/1_fusion_power_plant.png}
    \caption{Schematic diagram of nuclear fusion power plant}
    \label{fig:fusion_power_plant}
\end{figure}
}

\subsection{Explain the difference between magnetic and inertial confinement fusion.}
\solutionblock{Magnetic confinement fusion uses magnetic fields (e.g. tokamak) to confine the plasma and inertial confinement fusion uses lasers to heat up a pellet of fuel and then uses the inertia of the pellet to confine the plasma.\\}

\newpage
\section{Chapter 2: Discoveries Leading to Fusion}

\subsection{Give a historical perspective on the scientific discoveries that led to the discovery of fusion.}
\solutionblock{}

\subsection{Explain how the mass defect of various isotopes can be used to extract energy in fission and fusion.}
\solutionblock{}

\newpage
\section{Chapter 3: Stellar Energy and Fusion}

\subsection{Where does the energy in stars come from, and how was it conjectured?}
\solutionblock{Was conjectured by Sir Arthur Eddington in 1920s. He realized that the sun gains energy from the mass defect of Hydrogen to Helium.\\ If the sun was a lump of coal it would burn out in a few thousand years.\\ The sun must be older than the earth they argued and the earth is (according to Lord Kelvin) at least 20 million years old [Spoiler: he was was way off].\\}

\subsection{Write the reaction of the proton-proton cycle. Why is it important for stars?}
\solutionblock{
    1) Two protons combine to form deuterium (D) and a positron (e+) and a neutrino ($\nu$)\\
    2) The deuterium combines with another proton to form Helium-3 (He-3) and a gamma ray ($\gamma$)\\
    3) Two Helium-3 nuclei combine to form Helium-4 (He-4) and two protons.\\
    This process releases about 10 million times more energy than the chemical bonding of Hydrogen with Oxygen to form Water. $\rightarrow$ from that we conclude that the sun must live not for thousands, but for billions of years. \\
    \begin{figure}[H]
        \centering
        \includegraphics[width=0.8\textwidth]{chapters/fig/3_pp_cycle.png}
        \caption{Proton-Proton Cycle}
        \label{fig:proton_proton_cycle}
    \end{figure}
}

\subsection{How are stars able to generate conditions for fusion, and why do we need other ways on earth?}
\solutionblock{They are massive, so gravity compresses them and heats them up. They have to constantly balance the gravitational force with the pressure from the fusion reaction.\\
In the sun fusion only happens near the core - practically within a sphere 1/10 of the sun's radius. About 270 watts are produced per cubic meter in the core which is then transported through radiation and further out through convection.\\
In the core the sun has a temperature of about 14 million Kelvin while on the surface it is only about 6000 Kelvin.\\
On Earth we need other confinement methods because we need much higher energy densities than the sun.\\}

\subsection{What's the meaning of different star stages and their composition? What stage is the Sun and elements can it produce?}
\solutionblock{One might wonder how heavier elements than Helium are produced.\\
For bigger stars (than our sun) the fusion process continues after Helium-4 is produced. The star burns at a higher temperature and fuses Helium-4 to Carbon-12 and Carbon-12 to Oxygen-16, ... and so on Neon-20, Magnesium-24,...\\
Our sun can only produce Helium-4 and is subsequently doomed to cool down and shrink into a white dwarf.\\
Bigger stars explode in a supernova and produce all the heavier elements which are the building blocks for so called secondary stars (like our sun) and planets and \textit{us}.\\}

\subsection{Explain primordial nucleosynthesis and how it led to the current universe.}
\solutionblock{Also known as Big Bang Nucleosynthesis this refers to the production of nuclei other than Hydrogen during the early phases of the universe.\\ The universe was extremely hot and dense a few seconds after the big bang. As it expanded and cooled down the protons and neutrons combined to form Hydrogen, Helium-4 (pp-cycle) and a small amount of Lithium.\\ By pure change certain regions of the universe had a slightly higher density than others. These regions attracted more matter and eventually formed stars.\\}

\newpage
\section{Chapter 4: Fusion on Earth}

\subsection{Write equations and explain similarities and differences between fusion reactions that are realistically possible to do on earth.}
\solutionblock{Different reaction types: 
\begin{enumerate}
    \item D-T: \ch{D} + \ch{T} $\rightarrow$ \ch{^{4}He} + \ch{n} + 17.6 MeV
    \item D-D: \ch{D} + \ch{D} $\rightarrow$ \ch{^{3}He} + \ch{n} + 3.3 MeV
    \item D-D: \ch{D} + \ch{D} $\rightarrow$ \ch{T} + \ch{p} + 4.0 MeV
    \item D-Helium-3: \ch{D} + \ch{^{3}He} $\rightarrow$ \ch{^{4}He} + \ch{p} + 18.3 MeV
\end{enumerate}
The D-T reaction has the highest cross-section and can be accomplished with the lowest temperatures. The D-D reaction is the more difficult to achieve but doesn't require Tritium. Nevertheless, the most realistic to achieve is the D-T reaction.\\
}


\subsection{Why is tritium a scarce resource and how to produce it?}
\solutionblock{Because tritium is a radioactive isotope of Hydrogen with a half-life of 12.3 years. It is produced in the upper atmosphere by cosmic rays. Other than that there are no significant natural sources of tritium.\\
Tritium is therefore produced by irradiating Lithium-6 or Lithium-7 with neutrons: 
\begin{enumerate}
    \renewcommand\labelenumi{} 
    \item \ch{^{6}Li} + \ch{n} $\rightarrow$ \ch{T} + \ch{^{4}He} + 4.8 MeV
    \item \ch{^{7}Li} + \ch{n} $\rightarrow$ \ch{T} + \ch{^{4}He} + \ch{n} - 2.5 MeV
\end{enumerate}
Some would even be produced inside a fusion reactor by the D-D reaction.\\
}

\subsection{Draw the overall fuel cycle of a D-T fusion plant.}
\solutionblock{
    \begin{figure}[H]
        \centering
        \includegraphics[width=0.75\textwidth]{chapters/fig/4_fuel_cycle.png}
        \caption{Fuel Cycle of a D-T Fusion Plant}
        \label{fig:4_fuel_cycle}
    \end{figure}
}

\subsection{Explain the difficulty to achieve fusion regarding the Coulomb barrier. What makes it easier than an estimation via classical physics?}
\solutionblock{
For fusion to occur the Coulomb barrier has to be overcome. This is the repulsive force between the two nuclei due to their positive charge. However, if two nuclei get exeptionally close to each other the strong nuclear force takes over and binds them together releasing lots of energy in the process.\\
In classical physics the Coulomb barrier is so high that the probability of two nuclei getting close enough to fuse is practically zero. However, in quantum mechanics there is a small probability that the nuclei can tunnel through the barrier. This probability is dependent on the energy of the nuclei and thus on the temperature of the plasma. 
\begin{figure}[H]
    \centering
    \includegraphics[width=0.65\textwidth]{chapters/fig/4_coulomb.png}
    \caption{Coulomb Barrier}
    \label{fig:4_coulomb_barrier}
\end{figure}
}

\subsection{What temperatures are needed for thermonuclear fusion? Explain with regard to the reaction cross-section.}
\solutionblock{
    The D-T reaction has the highest cross section at about 100 keV. The cross section gives the probability of a reaction to occur. The higher the cross section the higher the probability. Each reaction has a characteristic cross section curve which peaks at a certain energy. The D-T reaction requires the lowest temperature to achieve fusion.\\
    \begin{figure}[H]
        \centering
        \includegraphics[width=0.65\textwidth]{chapters/fig/4_cross_section.png}
        \caption{Cross Section for different reactions}
        \label{fig:4_cross_section}
    \end{figure}
}

\subsection{Explain the power amplification factor, break-even point, and ignition criterion.}
\solutionblock{
\textbf{Power amplification factor: } A fusion energy gain factor, usually expressed with the symbol $Q$, is the ratio of fusion power produced in a nuclear fusion reactor to the power required to maintain the plasma in steady state. There are two break-even points: \textbf{engineering break-even} which is the point where the fusion power equals the heating power and \textbf{economic break-even} which is the point where the fusion power equals the heating power plus the power needed to produce the plasma i.e. operating costs of the plant.\\
Self heating of the plasma is only achieved when the gain factor is $Q\approx5$.\\
\textbf{Ignition criterion: } Ignition is the point where the fusion power produced is higher than the heating power. This is the point where the plasma is self-heating (menaing $Q \approx 5$).\\
}

\subsection{What is the fusion triple-product? Explain all three terms and the ways to get a high value in magnetic and inertial confinement fusion.}
\begin{multisolutionblock}
The triple product is a figure of merit used in fusion research. It is the product of the plasma density $n$, the plasma temperature $T$ and the energy confinement time $\tau_E$.\\
\begin{equation}
    nT\tau_E > 3 \cdot 10^{21} m^{-3}keV \, s
\end{equation}
\begin{figure}[H]
    \centering
    \includegraphics[width=0.65\textwidth]{chapters/fig/4_triple_product.png}
    \caption{The ignition criterion: the value of the product of density and confinement time $n\tau_E$, necessary to obtain plasma ignition, plotted as a function of plasma temperature $T$ (on x-axis). The curve has a minimum at about $T = 30 \,keV$ (roughly $300$ million K).}
    \label{fig:4_triple_product}
\end{figure}

For the two different types of confinement we can achieve a high triple product in different ways. For magnetic confinement we can achieve a somewhat high confinement time at a lower preasure. For inertial confinement we can achieve a high pressure at a lower confinement time.\\
For both cases a temperature $T$ of about 20-30 keV is required. A $Q$ of about 5, which is needed for ignition may be achieved using magnetic confinement at 5 bars for 1 second or inertial confinement at 5 billion bars for 1 nanosecond.\\
\begin{figure}[H]
    \centering
    \includegraphics[width=0.65\textwidth]{chapters/fig/4_inertial_vs_magnetic.png}
    \caption{The conditions required for fusion plotted in terms of plasma pressure (bars) against confinement time (in seconds).}
    \label{fig:4_inertial_vs_magnetic}
\end{figure}
\end{multisolutionblock}

\subsection{Discuss from a historical perspective how close we are to reach scientific and technical break-even and ignition for various fusion technologies.}
\solutionblock{From an engineering standpoint we are just now able to produce magnetic confinement devices that can achieve a high enough triple product to reach ignition. However, we are still far of when it comes to ignition an thus self heating. The constant of fusion research is that we are always 30 years away from a working fusion reactor.\\
Actually we are getting closer: recent experiments at the National Ignition Facility (NIF) in the US have achieved a triple product of $Q = 1.54$ which is a factor of about 3 shy of the ignition criterion.\\
For magnetic confinement the record is held by the Joint European Torus (JET) in the UK with a triple product of $Q = 0.67$. For $Q_{ext}$ (only measuring the energy input from external sources) the record stands at $Q_{ext} = 1.25$ slightly besting JET's $Q_{ext}$ of $1.14$.\\
}

\newpage
\section{Chapter 5: MISSING}
\newpage
\section{Chapter 6: The Hydrogen Bomb}

\subsection{Give a historical international perspective on the development of the H-bomb.}
\solutionblock{}

\subsection{Explain the requirement for a H-bomb and advantages over pure fission.}
\solutionblock{}

\subsection{Discuss limited resources and processing for Uranium and Tritium.}
\solutionblock{}

\subsection{How does an H-bomb in the Teller-Ulam design work? Draw and explain.}
\solutionblock{}

\subsection{Discuss ideas for civil uses of nuclear bombs and why they failed.}
\solutionblock{}

\newpage
\section{Chapter 7: MISSING}
\newpage
\section{Video Notes: }
\subsection{JET}
JET is a large collaborative project between many European countries. It is located in the UK and is the largest tokamak in the world. It is a research facility and has no goal of producing energy. It is used to test new technologies and materials for future fusion reactors.\\

JET does research to give input to ITER and DEMO. 
The wall of JET was changed to a beryllium and tungsten wall to test the material for ITER. It is the only tokamak in the world that can be fueled with DT.\\
Less fuel (tritium) trapped in the new wall. \\
Electron density is increased in for tritium rich plasmas, but the erosion of the wall is also increased by tritium.\\
Due to the new wall some Beryllium is present in the plasma, but this can be used to heat the plasma.\\

JET doubled the previous record for fusion power with a tritium rich plasma 59 MJ. 

SUMMARY: D-T fusion in ITER like conditions tested, we know more about burning plasma + models validated. 
\subsection{NIF - Lawrence Livermore National Lab}
Lawson criteria in inertial confinement fusion (principle is implosion):\\
They use indirect drive with a hohlraum -> the hohlraum surface emits x-rays which then heat the fuel pellet.\\
Typical pressure in X-rays on capsule surface is about 100-200 Mbar (in DT 100-550 Gbar).\\
Alpha heating: the alpha particles produced by the fusion reaction heat the plasma.\\
Implosion velocity is about 350 km/s and after the implosion the plasma explodes (taking energy out)\\
A burning ICF plasma means that the alpha heating is higher than the energy loss (by explotion, brems loss and conduction loss).\\
Changed design to get the alpha heating to exceed the energy loss and get to ignite. 

\subsection{MIT Plasma Sciene and Fusion Center}
\textbf{20 Tesla superconducting magnets} \\
High field means high gain: 
HTS (High Temperature Superconductors) > 4K -> operating space is bigger by a factor of a 1000\\
Tested - simulated and then tested in the lab. Approximately double the field strength of current magnets.-> to build SPARC as a proof of concept.\\
SPARC proofs that high field magnets are possible and that fusion reactors can run for about 10 seconds per pulse. 



\end{document}
